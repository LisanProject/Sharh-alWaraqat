% Options for packages loaded elsewhere
% Options for packages loaded elsewhere
\PassOptionsToPackage{unicode}{hyperref}
\PassOptionsToPackage{hyphens}{url}
\PassOptionsToPackage{dvipsnames,svgnames,x11names}{xcolor}
%
\documentclass[
  a4paper,
  DIV=11,
  numbers=noendperiod]{scrartcl}
\usepackage{xcolor}
\usepackage{amsmath,amssymb}
\setcounter{secnumdepth}{5}
\usepackage{iftex}
\ifPDFTeX
  \usepackage[T1]{fontenc}
  \usepackage[utf8]{inputenc}
  \usepackage{textcomp} % provide euro and other symbols
\else % if luatex or xetex
  \usepackage{unicode-math} % this also loads fontspec
  \defaultfontfeatures{Scale=MatchLowercase}
  \defaultfontfeatures[\rmfamily]{Ligatures=TeX,Scale=1}
\fi
\usepackage{lmodern}
\ifPDFTeX\else
  % xetex/luatex font selection
\fi
% Use upquote if available, for straight quotes in verbatim environments
\IfFileExists{upquote.sty}{\usepackage{upquote}}{}
\IfFileExists{microtype.sty}{% use microtype if available
  \usepackage[]{microtype}
  \UseMicrotypeSet[protrusion]{basicmath} % disable protrusion for tt fonts
}{}
\makeatletter
\@ifundefined{KOMAClassName}{% if non-KOMA class
  \IfFileExists{parskip.sty}{%
    \usepackage{parskip}
  }{% else
    \setlength{\parindent}{0pt}
    \setlength{\parskip}{6pt plus 2pt minus 1pt}}
}{% if KOMA class
  \KOMAoptions{parskip=half}}
\makeatother
% Make \paragraph and \subparagraph free-standing
\makeatletter
\ifx\paragraph\undefined\else
  \let\oldparagraph\paragraph
  \renewcommand{\paragraph}{
    \@ifstar
      \xxxParagraphStar
      \xxxParagraphNoStar
  }
  \newcommand{\xxxParagraphStar}[1]{\oldparagraph*{#1}\mbox{}}
  \newcommand{\xxxParagraphNoStar}[1]{\oldparagraph{#1}\mbox{}}
\fi
\ifx\subparagraph\undefined\else
  \let\oldsubparagraph\subparagraph
  \renewcommand{\subparagraph}{
    \@ifstar
      \xxxSubParagraphStar
      \xxxSubParagraphNoStar
  }
  \newcommand{\xxxSubParagraphStar}[1]{\oldsubparagraph*{#1}\mbox{}}
  \newcommand{\xxxSubParagraphNoStar}[1]{\oldsubparagraph{#1}\mbox{}}
\fi
\makeatother


\usepackage{longtable,booktabs,array}
\usepackage{calc} % for calculating minipage widths
% Correct order of tables after \paragraph or \subparagraph
\usepackage{etoolbox}
\makeatletter
\patchcmd\longtable{\par}{\if@noskipsec\mbox{}\fi\par}{}{}
\makeatother
% Allow footnotes in longtable head/foot
\IfFileExists{footnotehyper.sty}{\usepackage{footnotehyper}}{\usepackage{footnote}}
\makesavenoteenv{longtable}
\usepackage{graphicx}
\makeatletter
\newsavebox\pandoc@box
\newcommand*\pandocbounded[1]{% scales image to fit in text height/width
  \sbox\pandoc@box{#1}%
  \Gscale@div\@tempa{\textheight}{\dimexpr\ht\pandoc@box+\dp\pandoc@box\relax}%
  \Gscale@div\@tempb{\linewidth}{\wd\pandoc@box}%
  \ifdim\@tempb\p@<\@tempa\p@\let\@tempa\@tempb\fi% select the smaller of both
  \ifdim\@tempa\p@<\p@\scalebox{\@tempa}{\usebox\pandoc@box}%
  \else\usebox{\pandoc@box}%
  \fi%
}
% Set default figure placement to htbp
\def\fps@figure{htbp}
\makeatother





\setlength{\emergencystretch}{3em} % prevent overfull lines

\providecommand{\tightlist}{%
  \setlength{\itemsep}{0pt}\setlength{\parskip}{0pt}}



 


\KOMAoption{captions}{tableheading}
\makeatletter
\@ifpackageloaded{tcolorbox}{}{\usepackage[skins,breakable]{tcolorbox}}
\@ifpackageloaded{fontawesome5}{}{\usepackage{fontawesome5}}
\definecolor{quarto-callout-color}{HTML}{909090}
\definecolor{quarto-callout-note-color}{HTML}{0758E5}
\definecolor{quarto-callout-important-color}{HTML}{CC1914}
\definecolor{quarto-callout-warning-color}{HTML}{EB9113}
\definecolor{quarto-callout-tip-color}{HTML}{00A047}
\definecolor{quarto-callout-caution-color}{HTML}{FC5300}
\definecolor{quarto-callout-color-frame}{HTML}{acacac}
\definecolor{quarto-callout-note-color-frame}{HTML}{4582ec}
\definecolor{quarto-callout-important-color-frame}{HTML}{d9534f}
\definecolor{quarto-callout-warning-color-frame}{HTML}{f0ad4e}
\definecolor{quarto-callout-tip-color-frame}{HTML}{02b875}
\definecolor{quarto-callout-caution-color-frame}{HTML}{fd7e14}
\makeatother
\makeatletter
\@ifpackageloaded{caption}{}{\usepackage{caption}}
\AtBeginDocument{%
\ifdefined\contentsname
  \renewcommand*\contentsname{Table of contents}
\else
  \newcommand\contentsname{Table of contents}
\fi
\ifdefined\listfigurename
  \renewcommand*\listfigurename{List of Figures}
\else
  \newcommand\listfigurename{List of Figures}
\fi
\ifdefined\listtablename
  \renewcommand*\listtablename{List of Tables}
\else
  \newcommand\listtablename{List of Tables}
\fi
\ifdefined\figurename
  \renewcommand*\figurename{Figure}
\else
  \newcommand\figurename{Figure}
\fi
\ifdefined\tablename
  \renewcommand*\tablename{Table}
\else
  \newcommand\tablename{Table}
\fi
}
\@ifpackageloaded{float}{}{\usepackage{float}}
\floatstyle{ruled}
\@ifundefined{c@chapter}{\newfloat{codelisting}{h}{lop}}{\newfloat{codelisting}{h}{lop}[chapter]}
\floatname{codelisting}{Listing}
\newcommand*\listoflistings{\listof{codelisting}{List of Listings}}
\makeatother
\makeatletter
\makeatother
\makeatletter
\@ifpackageloaded{caption}{}{\usepackage{caption}}
\@ifpackageloaded{subcaption}{}{\usepackage{subcaption}}
\makeatother
\usepackage{bookmark}
\IfFileExists{xurl.sty}{\usepackage{xurl}}{} % add URL line breaks if available
\urlstyle{same}
\hypersetup{
  pdftitle={Commentary of ``Al-Waraqat'' by Imam Jalal al-Din al-Mahalli},
  pdfauthor={Liban Hussein},
  colorlinks=true,
  linkcolor={blue},
  filecolor={Maroon},
  citecolor={Blue},
  urlcolor={Blue},
  pdfcreator={LaTeX via pandoc}}


\title{Commentary of ``Al-Waraqat'' by Imam Jalal al-Din al-Mahalli}
\author{Liban Hussein}
\date{2025-11-09}
\begin{document}
\maketitle

\renewcommand*\contentsname{Table of contents}
{
\hypersetup{linkcolor=}
\setcounter{tocdepth}{3}
\tableofcontents
}

\section{Introduction}\label{sec-intro}

The long-term goal of this project is to provide an accessible,
organized study companion for English-speaking students of sacred law,
based on \textbf{Sharh al-Waraqat by Imam Jalal al-Din al-Mahalli}, and
taught according to reliable Shafi'i methodology and sources by
\textbf{Sheikh Abdullahi Shire (Miraath Institute)}
(\href{https://www.instagram.com/abdullahishire_/?hl=en}{Instagram}).

This work was originally inspired by a fellow student (may Allah
preserve him and protect him) who inquired about my note-taking approach
a while back. In the spirit of classical tradition, I was reminded of
\textbf{Ta'lim al-Muta'allim fi Tariq al-Ta'allum} (\emph{Instruction of
the Student: The Method of Learning}), which outlines adab and practical
methods for serious seekers of sacred knowledge.

\href{https://ia801504.us.archive.org/31/items/53330826InstructionOfTheStudentTaAlimAlMutaAllimTheMethodOfLearning/53330826-Instruction-of-the-Student-Ta-alim-al-Muta-allim-The-Method-of-Learning.pdf}{Read
the English translation of \emph{Ta'lim al-Muta'allim}}

\begin{center}\rule{0.5\linewidth}{0.5pt}\end{center}

\subsection{Organization}\label{organization}

The content follows the topical arrangement of \emph{Sharh al-Waraqat}
by Imam al-Mahalli, beginning with biographies of the author (Imam
al-Haramayn al-Juwayni) and the commentator (Imam Jalal al-Din
al-Mahalli), then progressing through the essential chapters of Usul
al-Fiqh, including, but not limited to categories of speech, rulings,
commands and prohibitions, general and specific texts, abrogation,
consensus, analogy (qiyas), and independent legal reasoning (ijtihad).

\begin{center}\rule{0.5\linewidth}{0.5pt}\end{center}

\subsection{How to Take Notes While Studying Islamic
Sciences}\label{note-taking}

Many students struggle to retain English explanations while following
the Arabic text, especially when encountering unfamiliar words and
concepts related to the subject matter at hand. Research from the
\href{https://weingartencenter.universitylife.upenn.edu/take-better-notes-the-relationship-between-time-and-memory/}{University
of Pennsylvania} shows that students remember \textbf{up to 70\% of
material when they write concise notes and review them within 24 hours},
compared to less than 30\% if no active note-taking or review is done.

\subsubsection{Preparation: Setting Up Your
Notes}\label{preparation-setting-up-your-notes}

\begin{tcolorbox}[enhanced jigsaw, toptitle=1mm, leftrule=.75mm, colback=white, left=2mm, arc=.35mm, opacityback=0, breakable, rightrule=.15mm, toprule=.15mm, coltitle=black, bottomtitle=1mm, titlerule=0mm, title=\textcolor{quarto-callout-tip-color}{\faLightbulb}\hspace{0.5em}{Before the Lesson}, opacitybacktitle=0.6, bottomrule=.15mm, colbacktitle=quarto-callout-tip-color!10!white, colframe=quarto-callout-tip-color-frame]

\begin{itemize}
\tightlist
\item
  Print or bring the Arabic text of \emph{Sharh al-Waraqat}. Lightly
  number each line or key statement in the matn.
\item
  Keep a separate notebook or tablet for your notes --- leave space
  beside each line number to write explanations.
\item
  Draw a narrow left column for keywords or questions and a larger right
  section for class explanations and commentary.
\item
  Leave a bottom section or back of the page for summarizing what you
  learned after class.
\end{itemize}

\end{tcolorbox}

\begin{center}\rule{0.5\linewidth}{0.5pt}\end{center}

\subsubsection{How to Structure Your Notes During
Class}\label{how-to-structure-your-notes-during-class}

Structure your \textbf{notebook page/notetaking device} like this:

\begin{longtable}[]{@{}
  >{\raggedright\arraybackslash}p{(\linewidth - 2\tabcolsep) * \real{0.5000}}
  >{\raggedright\arraybackslash}p{(\linewidth - 2\tabcolsep) * \real{0.5000}}@{}}
\toprule\noalign{}
\begin{minipage}[b]{\linewidth}\raggedright
Section
\end{minipage} & \begin{minipage}[b]{\linewidth}\raggedright
Purpose
\end{minipage} \\
\midrule\noalign{}
\endhead
\bottomrule\noalign{}
\endlastfoot
Left Margin (small) & Write Arabic line number, key term, or
question. \\
Main Notes Area (right side) & English explanation, teacher's
commentary, definitions, examples. \\
Bottom of Page / Separate Box & Summary of lesson, unresolved questions,
or reflections. \\
\end{longtable}

You may lightly underline important words in the Arabic text, but full
explanations should remain in the notebook or be referenced from
reliable translations. In addition, always, when referencing or quoting
a hadith by the Prophet Muhammad, always write out Salawat in it's full
form (صلى الله عليه وسلم).

\begin{center}\rule{0.5\linewidth}{0.5pt}\end{center}

\subsubsection{Minimal Color System}\label{minimal-color-system}

\begin{longtable}[]{@{}
  >{\raggedright\arraybackslash}p{(\linewidth - 2\tabcolsep) * \real{0.4375}}
  >{\raggedright\arraybackslash}p{(\linewidth - 2\tabcolsep) * \real{0.5625}}@{}}
\toprule\noalign{}
\begin{minipage}[b]{\linewidth}\raggedright
Color
\end{minipage} & \begin{minipage}[b]{\linewidth}\raggedright
Use For
\end{minipage} \\
\midrule\noalign{}
\endhead
\bottomrule\noalign{}
\endlastfoot
Blue & Key Arabic terms \& definitions. \\
Green & Important commentary or explanations from Sheikh Abdullahi
Shire. \\
Red & Questions, uncertainties, or topics to review or ask about. \\
\end{longtable}

\begin{center}\rule{0.5\linewidth}{0.5pt}\end{center}

\subsubsection{Evidence-Based Review
Schedule}\label{evidence-based-review-schedule}

Research shows that reviewing notes at spaced intervals drastically
increases retention. A simple routine I've found to be helpful is:

\begin{longtable}[]{@{}
  >{\raggedright\arraybackslash}p{(\linewidth - 4\tabcolsep) * \real{0.2400}}
  >{\raggedright\arraybackslash}p{(\linewidth - 4\tabcolsep) * \real{0.5200}}
  >{\raggedright\arraybackslash}p{(\linewidth - 4\tabcolsep) * \real{0.2400}}@{}}
\toprule\noalign{}
\begin{minipage}[b]{\linewidth}\raggedright
Time
\end{minipage} & \begin{minipage}[b]{\linewidth}\raggedright
What to Do
\end{minipage} & \begin{minipage}[b]{\linewidth}\raggedright
Why?
\end{minipage} \\
\midrule\noalign{}
\endhead
\bottomrule\noalign{}
\endlastfoot
Within 24-48 hours & Reread your notes and add missing details (this is
especially important if you are re-watching the lectures and catch
important missed information) & This prevents forgetting up to 50--80\%
of material. \\
After 3-6 days & Revisit and summarize the key ideas in your own words.
If you're able to, explain to other students in the class. & This not
only reinforces your own understanding, but helps other students benefit
as well and helps build lasting relationships with students you may take
subsequent classes with. \\
\end{longtable}

\begin{center}\rule{0.5\linewidth}{0.5pt}\end{center}

\section{Biographies of the Authors}\label{biographies-of-the-authors}

\begin{tcolorbox}[enhanced jigsaw, toptitle=1mm, leftrule=.75mm, colback=white, left=2mm, arc=.35mm, opacityback=0, breakable, rightrule=.15mm, toprule=.15mm, coltitle=black, bottomtitle=1mm, titlerule=0mm, title=\textcolor{quarto-callout-note-color}{\faInfo}\hspace{0.5em}{Source}, opacitybacktitle=0.6, bottomrule=.15mm, colbacktitle=quarto-callout-note-color!10!white, colframe=quarto-callout-note-color-frame]

Imam al-Juwayni's biography was translated from from \emph{Sharh
al-Waraqat}, Dar al-Diyaʿ Publishing, pages 13--15, and Imam Jalal
al-Din al-Mahalli's from the same text, pages 16-18.

\end{tcolorbox}

\subsection{Biography of Imam al-Haramain al-Juwayni (419--478 AH /
1028--1085
CE)}\label{biography-of-imam-al-haramain-al-juwayni-419478-ah-10281085-ce}

\subsubsection{Birth and Background}\label{birth-and-background}

He is \textbf{Shaykh al-Islam}, the jurist, theologian, and master of
usul, the Shaykh of the Shafi`is, the Imam of the Imams --- \textbf{Imam
al-Haramain Abu al-Ma`ali Abd al-Malik ibn Abd Allah ibn Yusuf
al-Juwayni al-Naysaburi}.

He was born into a well-known scholarly family among the people of
knowledge, virtue, piety, and asceticism.

His father, \textbf{Abu Muhammad Abd Allah al-Juwayni}, was among the
great imams of Islam. The son grew up under his care, acquiring
knowledge from him, memorizing, studying, and refining his worship. He
advanced in \textbf{fiqh}, \textbf{usul}, \textbf{tafsir}, and
\textbf{adab}.

\begin{itemize}
\tightlist
\item
  \textbf{Grandfather:} Yusuf ibn Abd Allah, a distinguished man of
  letters.\\
\item
  \textbf{Paternal uncle:} Abu al-Hasan Ali ibn Yusuf al-Juwayni, known
  as \emph{the Shaykh of the Hijaz}.\\
\item
  His era was among the most brilliant centuries; scholars flourished in
  every discipline and the intellectual climate refined his scholarship.
\end{itemize}

\subsubsection{His Teachers}\label{his-teachers}

He sought knowledge tirelessly with many leading scholars:

\begin{itemize}
\tightlist
\item
  \textbf{His father}, Imam Abu Muhammad al-Juwayni.\\
\item
  \textbf{Abu al-Qasim al-Iskafi al-Isfarayini}, learned jurist in
  usul.\\
\item
  \textbf{Abu al-Hasan Ali ibn Faddal al-Majjashi`i}, grammarian and
  literatus.\\
  He also heard hadith from \textbf{Abu Hasan}, \textbf{Abu Sa`id ibn
  'Alayyak}, and \textbf{Abu Sa`id al-Nasrawi}, among others.
\end{itemize}

\subsubsection{His Students}\label{his-students}

He taught numerous eminent figures, including:

\begin{itemize}
\tightlist
\item
  \textbf{Abu Hamid al-Ghazali (Hujjat al-Islam)}, author of \emph{Ihya`
  'Ulum al-Din}.\\
\item
  \textbf{Imad al-Din Abu al-Hasan Ali ibn Muhammad al-Harrasi}, author
  of valuable works.\\
\item
  The jurist \textbf{Ahmad ibn Muhammad al-Khawwafi al-Naysaburi} (a
  contemporary).
\end{itemize}

\begin{tcolorbox}[enhanced jigsaw, toptitle=1mm, leftrule=.75mm, colback=white, left=2mm, arc=.35mm, opacityback=0, breakable, rightrule=.15mm, toprule=.15mm, coltitle=black, bottomtitle=1mm, titlerule=0mm, title=\textcolor{quarto-callout-tip-color}{\faLightbulb}\hspace{0.5em}{Praise for His Students}, opacitybacktitle=0.6, bottomrule=.15mm, colbacktitle=quarto-callout-tip-color!10!white, colframe=quarto-callout-tip-color-frame]

Imam al-Juwayni said:\\
\textgreater{} ``Al-Ghazali is a deep ocean, al-Kiya is a blazing lion,
and al-Khawwafi is a consuming fire.''

\end{tcolorbox}

\begin{tcolorbox}[enhanced jigsaw, toptitle=1mm, leftrule=.75mm, colback=white, left=2mm, arc=.35mm, opacityback=0, breakable, rightrule=.15mm, toprule=.15mm, coltitle=black, bottomtitle=1mm, titlerule=0mm, title=\textcolor{quarto-callout-note-color}{\faInfo}\hspace{0.5em}{Praise from Scholars}, opacitybacktitle=0.6, bottomrule=.15mm, colbacktitle=quarto-callout-note-color!10!white, colframe=quarto-callout-note-color-frame]

\begin{itemize}
\tightlist
\item
  \textbf{Al-Dhahabi}: \emph{``The great Imam, Shaykh of the
  Shafi`iyyah, Imam al-Haramain.''}\\
\item
  \textbf{Al-Subki}: \emph{``Imam, Shaykh of Islam, sea of knowledge,
  precise scholar, theologian and usuli, eloquent and unmatched ---
  famed East and West.''}\\
\item
  \textbf{Abu Uthman al-Sabuni}: \emph{``He is the very eye of Islam
  today, defending it with eloquence.''}
\end{itemize}

\end{tcolorbox}

\subsubsection{His Works}\label{his-works}

\begin{itemize}
\tightlist
\item
  \emph{Nihayat al-Matlab fi Dirayat al-Madhhab}\\
\item
  \emph{Al-Burhan fi Usul al-Fiqh}\\
\item
  \emph{Al-Irshad ila Qawati` al-Adillah fi Usul al-I`tiqad}\\
\item
  \emph{Al-Waraqat fi Usul al-Fiqh}\\
\item
  \emph{Luma` al-Adillah}\\
\item
  \emph{Ghayat al-Umam fi Ithbat al-'Adl li al-Hakam}\\
  \ldots and others.
\end{itemize}

\subsubsection{His Death}\label{his-death}

He passed away in \textbf{478 AH (1085 CE)}. May Allah envelop him in
His mercy, Allahumma ameen.

\begin{center}\rule{0.5\linewidth}{0.5pt}\end{center}

\subsection{Biography of Imam Jalal al-Din al-Mahalli (791--864 AH /
1389--1459
CE)}\label{biography-of-imam-jalal-al-din-al-mahalli-791864-ah-13891459-ce}

\subsubsection{Birth and Background}\label{birth-and-background-1}

He is the learned, precise imam --- hadith scholar, jurist, theologian,
exegete, and grammarian ---\\
\textbf{Jalal al-Din Muhammad ibn Ahmad ibn Muhammad al-Ansari
al-Mahalli al-Qahiri al-Shafi`i}.

Born in \textbf{Cairo} in \textbf{791 AH (1389 CE)}, he dedicated
himself to both transmitted and rational sciences and excelled in most
of what he studied.

\subsubsection{His Teachers}\label{his-teachers-1}

He studied under many great imams:

\begin{itemize}
\tightlist
\item
  \textbf{Siraj al-Din Abu Hafs 'Umar ibn 'Ali (Ibn al-Mulqin)}\\
\item
  \textbf{Shaykh al-Islam al-Bulqini} (Abu Hafs 'Umar ibn Raslan
  al-Qahiri al-Shafi`i)\\
\item
  \textbf{Wali al-Din Abu Zur`ah Ahmad ibn 'Abd al-Raheem al-'Iraqi}\\
\item
  \textbf{Shihab al-Din Ahmad ibn 'Ali al-Kinani (Ibn Hajar
  al-'Asqalani)} --- Commander of the Faithful in Hadith
\end{itemize}

\subsubsection{His Students}\label{his-students-1}

Students who graduated from his teaching include:

\begin{itemize}
\tightlist
\item
  \textbf{Abu al-Fida' Isma`il ibn Ibrahim (Ibn Jama`ah)}\\
\item
  \textbf{Abu 'Abd Allah Muhammad ibn 'Abd al-Rahman al-Dimashqi (Ibn
  Qadi 'Ajlun)}\\
\item
  \textbf{Shams al-Din Muhammad ibn 'Abd al-Rahman al-Sakhawi}\\
\item
  \textbf{Jalal al-Din al-Suyuti}\\
\item
  \textbf{Abu al-Hasan 'Ali ibn 'Abd Allah al-Sharif al-Samhudi} ---
  scholar of Madinah
\end{itemize}

\begin{tcolorbox}[enhanced jigsaw, toptitle=1mm, leftrule=.75mm, colback=white, left=2mm, arc=.35mm, opacityback=0, breakable, rightrule=.15mm, toprule=.15mm, coltitle=black, bottomtitle=1mm, titlerule=0mm, title=\textcolor{quarto-callout-note-color}{\faInfo}\hspace{0.5em}{Praise from Scholars}, opacitybacktitle=0.6, bottomrule=.15mm, colbacktitle=quarto-callout-note-color!10!white, colframe=quarto-callout-note-color-frame]

\begin{itemize}
\tightlist
\item
  \textbf{Al-Sakhawi}: \emph{``Precise, sharp-minded, careful with his
  time, ascetic in spirit, and the Sheikh of the Shafi`is at the
  Mu'ayyadiyya and Barquqiyya institutions; an imam, verifier, deeply
  insightful and intelligent.''}\\
\item
  \textbf{Al-Suyuti}: \emph{``A sign of intelligence and
  understanding.''}\\
\item
  \textbf{Ibn al-Imad}: \emph{``The Taftazani of the Arabs --- the imam,
  the learned scholar.''}
\end{itemize}

\end{tcolorbox}

\subsubsection{His Works}\label{his-works-1}

\begin{itemize}
\tightlist
\item
  \emph{Al-Badr al-Tali` fi Hall Jam` al-Jawami` (Commentary on Subki's
  Jam' al-Jawami' on Usul al Fiqh)}\\
\item
  \emph{Sharh al-Waraqat} (this commentary)\\
\item
  \emph{Kanz al-Raghibin fi Sharh Minhaj al-Talibin (Commentary on Imam
  Nawawi's Minhaj in Fiqh)}\\
  \ldots and others.
\end{itemize}

\subsubsection{His Death}\label{his-death-1}

He passed away in \textbf{864 AH (1459 CE)}. May Allah have abundant
mercy on him. Allahumma ameen.




\end{document}
